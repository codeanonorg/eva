\documentclass[11pt,twoside,french]{article}
\usepackage{beraserif}
\usepackage{berasans}
\usepackage{beramono}
\usepackage{eulervm}
\usepackage[T1]{fontenc}
\usepackage[utf8]{inputenc}
% \usepackage[latin1]{inputenc}
\usepackage[a4paper]{geometry}
\geometry{verbose,lmargin=5cm,rmargin=4cm}
\usepackage{color}
\usepackage{babel}
\makeatletter
\addto\extrasfrench{%
   \providecommand{\og}{\leavevmode\flqq~}%
   \providecommand{\fg}{\ifdim\lastskip>\z@\unskip\fi~\frqq}%
}

\newcommand{\noun}[1]{\textsc{#1}}
\makeatother
\usepackage{refstyle}
\usepackage{algorithm2e}
\usepackage{microtype}
\usepackage[unicode=true,pdfusetitle,
 bookmarks=true,bookmarksnumbered=true,bookmarksopen=false,
 breaklinks=true,pdfborder={0 0 0},pdfborderstyle={},backref=section,colorlinks=true]
 {hyperref}
\hypersetup{
 pdfsubject={Un assembleur et VM pour les étudiants, par des étudiants},
 pdfkeywords={asm,assembleur,vm}}

\makeatletter

\begin{document}
\title{EVA}
\author{Association CodeAnon}

\maketitle
\clearpage{}

\tableofcontents{}

\cleardoublepage{}

\vfill{}


\section{Introduction}

Le langage assembleur constitue la première abstraction à la programmation
de circuits programmables. Cela rend l'assembleur facile à apprendre
et manipuler, mais difficile à utiliser et maîtriser. Face à l'état
des choses, nous avons décidé de développer \noun{Eva}.

Eva est une machine virtuelle développée par des étudiants pour les
étudiants. Simple et légère, elle offre un terrain d'expérimentation
particulièrement adapté à l'apprentissage de l'assembleur ou encore
à l'écriture de compilateurs. Son code source publique, clair et documenté
la rendent accessible et facilement modifiable. Eva est programmable
dans un langage assembleur dédié facile d'accès et doté d'un jeu d'instructions
succinct mais complet. Par ailleurs, l'implémentation d'Eva ainsi
que de son assembleur est accessible et reproductible facilement dans
n'importe quel langage!

\vfill{}
\newpage{}

\section{Éléments de conception}

La machine virtuelle doit être minimaliste pour être facilement implantable,
simple à programmer pour rester accessible aux étudiants désireux
d'apprendre, mais également suffisament performante pour supporter
le développement d'applications en vrai grandeur. \noun{Eva} étant
conçue sur la base du \og retour aux fondamentaux\fg , elle va à l'essentiel
et propose un jeu d'instructions à la fois concis et simple à prendre
en main sans fonctionnalités superflue. Nous avons à ces fins séléctionné
un sous-ensemble des instructions ARM \tabref{opcodes}.
\begin{table}[bp]
\begin{centering}
\begin{tabular}{|c|l|}
\hline
\textbf{Instruction} & \textbf{Description}\tabularnewline
\hline
\hline
\textbf{ADD} & Addition\tabularnewline
\hline
\textbf{ADC} & Addition avec retenue\tabularnewline
\hline
\textbf{BIC} & RAZ de bit\tabularnewline
\hline
\textbf{CMP} & Comparaison\tabularnewline
\hline
\textbf{MOV} & Écriture de valeur dans le registre\tabularnewline
\hline
\textbf{B} & \emph{Branch Jump}\tabularnewline
\hline
\textbf{BL} & \emph{Branch Link}\tabularnewline
\hline
\textbf{PUSH} & Empiler\tabularnewline
\hline
\textbf{POP} & Dépiler\tabularnewline
\hline
\textbf{SUB} & Soustration avec retenue\tabularnewline
\hline
\end{tabular}
\par\end{centering}
\caption{\label{tab:opcodes}Liste des op-codes de la machine virtuelle d'\noun{Eva}.}
\end{table}


\section{Implantation de la machine virtuelle}

Afin d'aider au développement du projet, \noun{Eva} va être implantée
en C (norme 99) et doit ne dépendre d'aucun code source autre que
la librairie standard. Une telle contrainte assure que l'implantation
soit élémentaire, le code source portable, et la compilation aisée.

Cette implantation minimale va de paire avec l'idéologie derrière
Eva : prioriser l'accessibilité et la fiabilité. De plus, l'aspect
sécurité est d'une grande importance pour le projet, où la correction
de code aura toujours priorité sur l'ajout de nouvelles fonctionnalités.

\section{Disponibilité du projet}

\noun{Eva} est un projet publique, dont le code et les binaires seront
accessibles en open-source, suivant une licence MIT. Cela veut dire
que n'importe quel utilisateur peut télécharger, lire et modifier
le code source. Le projet étant pédagogique, il va de soit que chaque
facette du projet soit publique et disponible au plus grand nombre.

\section{Projets connexes}

En plus du développement de la machine virtuelle, \noun{Eva} sera
également le socle de plusieurs projets dont divers compilateurs ciblant
\noun{Eva}. Parmis ces compilateurs l'équipe d'\noun{Eva} développera et maintiendra
un langage de programmation de haut niveau spécifique à \noun{Eva}. Ce dernier
permetra de complémenter les possibilités offertes par le simple langage
assembleur supporté nativement par la platteforme.
\end{document}
